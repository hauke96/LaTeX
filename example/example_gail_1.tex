\documentclass[a4paper]{article}
\usepackage{amsmath}
\usepackage{amsfonts}
\usepackage{amssymb}
\usepackage[ngerman]{babel}  
\usepackage[utf8]{inputenc}

\usepackage{../gail}
\usepackage{../dadp}

\setfirstauthor{Max Mustermann}
\setfirstauthorID{max@mustermann.de}
\setsecondauthor{Theo Retiker}
\setsecondauthorID{info@retiker-studios.com}

% this shows a dummy error message
%\setthirdauthor{Foo Bar} % MISSING!
\setthirdauthorID{foo@bar.org}

% There's a maximum amount of four authors
\setfourthauthor{Hauke Stieler}
\setfourthauthorID{mail@hauke-stieler.de}

\settitle{Introducing NFCs as authentification protocoll}
\setsheetnumber{3}
\setstartdate{2016}{04}{01}
\setdatefreq{7}
\setinterruptions{2}

\setsectionstyletasksalphnum{}

\begin{document}
	\maketitle
	\section{}
		This is a test file with some sections in it. Every section contains the sheetnumber as first number. Use \texttt{\textbackslash setsectionstyleplain\{\}}.
		\subsection{One of our subsections}
			This is an example subsection.
			\subsubsection{}
			There are also subsubsections ;)
	\section{How the dates work}
		The command \texttt{\textbackslash setstartdate\{YEAR\}\{MONTH\}\{DAY\}} sets the start date, but this is not the first release date. The first release Date is the date one interval later.\\
		The interval \texttt{\textbackslash setdatefreq\{DAYS\}} defines the frequency of the releases. This is used for the date calculation.\\
		If there're some release interrupts (e.g. due to hollydays), the \texttt{\textbackslash setinterruptions\{AMOUNT\}} is your friend.
\end{document}